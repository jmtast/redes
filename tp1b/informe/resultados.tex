\section{Resultados}

\subsection{RTT Teórico}

En este apartado describimos el desarrollo del cálculo teórico del \textit{Roundtrip Time} (RTT).
\\
Sabemos que el RTT es el tiempo que tarda la información en ir y volver, es decir:

\begin{equation}
	Rtt = 2 * Delay
\end{equation}

Es por esto que necesitamos calcular el delay, o tiempo que tarda la información en llegar al destino desde una fuente dada:

\begin{equation}
	Delay = T_{tx} + T_{prop}
\end{equation}

Creemos acorde para este caso, considerar despreciable el tiempo de transferencia o llenado del medio $T_{tx}$, dado que en comparación al tiempo de propagación, es ínfimo. Entonces debemos calcular:

\begin{equation}
	T_{prop} = D/V
\end{equation}

con $D$ la distancia del enlace y $V$ la velocidad de propagación. La distancia del enlace varía según el caso probado, y la velocidad de propagación la sacamos del enunciado ($2*10^5$ km/s).\\
\\
Distancias:
\begin{itemize}
	\item Universidad de California, Berkeley: \textbf{10412 km}
	\item Universidad de Oxford: \textbf{11107 km}
	\item Universidad de Tokio: \textbf{18372 km}
\end{itemize}

Tiempos de propagación (D/V):
\begin{itemize}
	\item Universidad de California, Berkeley: \textbf{0.05206 s}
	\item Universidad de Oxford: \textbf{0.055535 s}
	\item Universidad de Tokio:  \textbf{0.09186 s}
\end{itemize}

Al haber considerado despreciable el $T_{tx}$, el delay nos queda igual que el tiempo de propagación. Luego:\\
\\
Roundtrip Time ($2 * delay$):
\begin{itemize}
	\item Universidad de California, Berkeley: \textbf{0.10412 s}
	\item Universidad de Oxford: \textbf{0.11107 s}
	\item Universidad de Tokio:  \textbf{0.18372 s}
\end{itemize}