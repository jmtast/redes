\section{Resultados}

\subsection{RTT Teórico}

En este apartado describimos el desarrollo del cálculo teórico del \textit{Roundtrip Time} (RTT).
\\
Sabemos que el RTT es el tiempo que tarda la información en ir y volver, es decir:

\begin{equation}
	Rtt = 2 * Delay
\end{equation}

Es por esto que necesitamos calcular el delay, o tiempo que tarda la información en llegar al destino desde una fuente dada:

\begin{equation}
	Delay = T_{tx} + T_{prop}
\end{equation}

Creemos acorde para este caso, considerar despreciable el tiempo de transferencia o llenado del medio $T_{tx}$, dado que en comparación al tiempo de propagación, es ínfimo. Entonces debemos calcular:

\begin{equation}
	T_{prop} = D/V
\end{equation}

con $D$ la distancia del enlace y $V$ la velocidad de propagación. La distancia del enlace varía según el caso probado, y la velocidad de propagación la sacamos del enunciado ($2*10^5$ km/s). Dados los cálculos anteriores, los resultados son los siguientes:\\

\begin{center}
\begin{tabular}{| c | c | c | c |} \hline
	Universidad		&	\textbf{Berkeley}		&	\textbf{Oxford}		&	\textbf{Tokio}		\\ \hline
	Distancia		&	10412 km		&	11107 km		&	18372 km		\\ \hline
	$T_{prop}$		&	52.06 ms	&	55.535 ms	&	91.86 ms	\\ \hline
	Delay			&	52.06 ms	&	55.535 ms	&	91.86 ms	\\ \hline
	\textbf{Roundtrip Time}	&	\textbf{104.12 ms}	&	\textbf{111.07 ms}	&	\textbf{183.72 ms}	\\ \hline
\end{tabular}
\end{center}

\subsection{RTT Empírico}

En las siguientes figuras se presentan los gráficos y diagramas realizados en función de las muestras de RTT realizadas con las distintas implementaciones de traceroute y los experimentos propuestos. Las franjas horarias en que se experimentó son \textsl{00hs, 02hs, 08hs, 11hs, 15hs, 22hs}. Es importante aclarar que las mismas corresponden a un día laboral y al huso horario de la ciudad origen, es decir GMT-3 según se utiliza en la Ciudad Autónoma de Buenos Aires. En caso de que la franja represente otro huso horario, la aclaración pertinente se encontrará en el análisis correspondiente al experimento. Además, en los gráficos en función de las franjas, los valores de RTT han sido promediados de un total de 20 ejecuciones de la herramienta.

\begin{figure}[h!]
  \centering
  \includegraphics[width=0.6\textwidth]{./figs/franjasTR_py_vs_linux.png}
  \caption{RTTs de ambos traceroute a cada destino en función de franja horaria}
  \label{fig:frajasTR_py_lin}
\end{figure}

\clearpage

\begin{figure}[h!]
  \centering
  \includegraphics[width=0.7\textwidth]{./figs/rtt_emp_vs_teo_berkeley.png}
  \caption{RTTs de ambos traceroute a Berkeley en contraste con el cálculo teórico}
  \label{fig:emp_vs_teo_berk}
\end{figure}

\begin{figure}[h!]
  \centering
  \includegraphics[width=0.7\textwidth]{./figs/rtt_emp_vs_teo_oxford.png}
  \caption{RTTs de ambos traceroute a Oxford en contraste con el cálculo teórico}
  \label{fig:emp_vs_teo_ox}
\end{figure}

\begin{figure}[h!]
  \centering
  \includegraphics[width=0.7\textwidth]{./figs/rtt_emp_vs_teo_tokyo.png}
  \caption{RTTs de ambos traceroute a Tokyo en contraste con el cálculo teórico}
  \label{fig:emp_vs_teo_tok}
\end{figure}

\begin{figure}[h!]
  \centering
  \includegraphics[width=0.7\textwidth]{./figs/rtt_horarios_california.png}
  \caption{RTT representativo de la Universidad de Berkeley, California}
  \label{fig:rtt-horarios-california}
\end{figure}

\begin{figure}[h!]
  \centering
  \includegraphics[width=0.7\textwidth]{./figs/rtt_horarios_oxford.png}
  \caption{RTT representativo de la Universidad de Oxford}
  \label{fig:rtt-horarios-oxford}
\end{figure}

\begin{figure}[h!]
  \centering
  \includegraphics[width=0.7\textwidth]{./figs/rtt_horarios_tokyo.png}
  \caption{RTT representativo de la Universidad de Tokio}
  \label{fig:rtt-horarios-tokyo}
\end{figure}

\clearpage

\begin{figure}[t!]
  \centering
  \includegraphics[width=0.7\textwidth]{./figs/rtt_normalizado.png}
  \caption{RTT a las 3 universidades en fase según un día laboral global}
  \label{fig:rtt-normalizado}
\end{figure}

\subsection{T$_{queue}$}

El $T_{queue}$ es el tiempo que permanece un paquete en la cola de cada hop, hasta que viaja al siguiente destino. Calculamos el $T_{queue}$ global de cada destino y algunos $T_{queue}$ específicos de los hops intermedios (los que pudimos, dada la información que nos consiguió brindar traceroute). Los resultados son los siguientes\footnote{La distancia entre hops se calculó tomando sus IPs, obteniendo las coordenadas y estimando distancias con http://boulter.com/gps/distance/}:\\

\begin{center}
\begin{tabular}{| c | c | c |}	\hline
	Berkeley 	& $T_{queue}$	& Distancia al anterior 	\\ \hline
	Hop 1	& 27.070		& 9939.91 km			\\ \hline
	Hop 2	& 50.680		& 4807.03 km		\\ \hline
	Hop 3	& 55.176		& 7221.12 km 	\\ \hline
	Hop 4	& 75.702		& 9558.47 km 	\\ \hline
	Hop 5	& 31.973		& 9062.53 km	 \\ \hline
	Hop 6	& 56.413		& 520.43 km	\\ \hline
	Promedio & 18.238	& -	\\ \hline
	Global	& 237.097	& -	\\ \hline
\end{tabular}
\vspace{1cm}
\begin{tabular}{| c | c | c |}	\hline
	Oxford 	& $T_{queue}$	& Distancia al anterior 	\\ \hline
	Hop 1	& 51.158		& 4491.43 km			\\ \hline
	Hop 2	& 53.750		& 486.91 km		\\ \hline
	Hop 3	& 131.08		& 4807.03 km 	\\ \hline
	Hop 4	& 5.3286		& 5708.73 km 	\\ \hline
	Hop 5	& 374.57 	& 360.94 km	 \\ \hline
	Hop 6	& 3.2951 	& 281.98 km	\\ \hline
	Promedio & 36.931	& -	\\ \hline
	Global	& 553.975	& -	\\ \hline
\end{tabular}
\vspace{1cm}
\begin{tabular}{| c | c | c |}	\hline
	Tokio 	& $T_{queue}$	& Distancia al anterior 	\\ \hline
	Hop 1	& 40.831 	& 9939.91 km			\\ \hline
	Hop 2	& 7.8821		& 7608.13 km		\\ \hline
	Hop 3	& 52.497		& 3329.60 km 	\\ \hline
	Hop 4	& 29.638		& 6656.10 km 	\\ \hline
	Promedio & 71.799	& -	\\ \hline
	Global	& 933.396	& -	\\ \hline
\end{tabular}
\end{center}
