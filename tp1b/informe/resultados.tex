\section{Resultados}

\subsection{RTT Teórico}

En este apartado describimos el desarrollo del cálculo teórico del \textit{Roundtrip Time} (RTT).
\\
Sabemos que el RTT es el tiempo que tarda la información en ir y volver, es decir:

\begin{equation}
	Rtt = 2 * Delay
\end{equation}

Es por esto que necesitamos calcular el delay, o tiempo que tarda la información en llegar al destino desde una fuente dada:

\begin{equation}
	Delay = T_{tx} + T_{prop}
\end{equation}

Creemos acorde para este caso, considerar despreciable el tiempo de transferencia o llenado del medio $T_{tx}$, dado que en comparación al tiempo de propagación, es ínfimo. Entonces debemos calcular:

\begin{equation}
	T_{prop} = D/V
\end{equation}

con $D$ la distancia del enlace y $V$ la velocidad de propagación. La distancia del enlace varía según el caso probado, y la velocidad de propagación la sacamos del enunciado ($2*10^5$ km/s). Dados los cálculos anteriores, los resultados son los siguientes:\\

\begin{center}
\begin{tabular}{| c | c | c | c |} \hline
	Universidad		&	\textbf{Berkeley}		&	\textbf{Oxford}		&	\textbf{Tokio}		\\ \hline
	Distancia		&	10412 km		&	11107 km		&	18372 km		\\ \hline
	$T_{prop}$		&	52.06 ms	&	55.535 ms	&	91.86 ms	\\ \hline
	Delay			&	52.06 ms	&	55.535 ms	&	91.86 ms	\\ \hline
	\textbf{Roundtrip Time}	&	\textbf{104.12 ms}	&	\textbf{111.07 ms}	&	\textbf{183.72 ms}	\\ \hline
\end{tabular}
\end{center}
