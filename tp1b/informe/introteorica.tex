\section{Introducción teórica}

En el presente trabajo vamos a experimentar con la capa tres del modelo OSI,
es decir, con el nivel de red.

En la primera parte vamos a implementar la herramienta \textit{ping} sobre el protocolo
ICMP. Una vez programada dicha herramienta, vamos a utilizarla para hacer
estimaciones del RTT a distintas partes del mundo. Para lograr una medición
más precisa haremos un promedio de $n$ pruebas para evitar que otros factores
influyan demasiado (como la congestión de la red, entre otros). Luego para
cada host analizado se realiza una comparación con el RTT teórico, asumiendo
una distancia lineal al objetivo y el uso de fibra óptica como único medio
físico.
Luego implementaremos la herramienta \textit{traceroute}, muy usada
para el análisis del viaje de paquetes a través de la red, aprovechando el
algoritmo implementado en la primera parte. Luego se harán pruebas a distintos
destinos y en diferentes momentos del día para obtener un mejor análisis del
estado de las rutas en la red.
