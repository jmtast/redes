\section{Conclusiones}

La experimentación desarrollada en este trabajo permitió confrontar dos paradigmas de manera muy clara: el teórico y el empírico, y poder realizar ricas y precisas observaciones sobre el comportamiento de las comunicaciones a nivel de red. En primer término se puede observar que el modelado teórico del RTT deja muchos factores cruciales de lado, motivo por el cual es razonable que la experimentación concreta haya arrojado valores tan disímiles tanto entre sí como frente al teórico. En el desplazamiento de paquetes en las redes no sólo se aplican las leyes de la física, sino también las culturales de la esfera humana de la que estas comunicaciones son fruto. Por esto se comprobó que en cierta medida también resulta importante tener en cuenta el momento en que se desarrolla una comunicación y qué tipo de actividades se están llevando a cabo, en principio en los extremos de los enlaces, y luego en los hops intermedios que conforman la ruta. Si bien puede anticiparse que en horarios nocturnos o de madrugada la latencia sea menor a la del día, en algunos casos se vio que no necesariamente esta idea se condice con la realidad.\\
\indent Otra conclusión importante que emana del trabajo aquí desarrollado es que también resulta dificil poder modelar experimentos utilizando, por ejemplo, el algoritmo de traceroute mediante ICMP, ya que por cada iteración en que se aumenta el $ttl$ se realiza una nueva comunicación que $a$ $priori$ poco tiene que ver con la anterior (distintas condiciones de la ruta o mismo porciones de ruta distintas, puede no haber respuestas, etc). Esto complica las predicciones e introduce un alto grado de variabilidad en los resultados recolectados.
