\section{Análisis}

En la \textbf{Figura~\ref{fig:frajasTR_py_lin}} se exhiben de manera general los resultados de RTT obtenidos luego de haber corrido tanto el traceroute propio como el de linux hacia las 3 universidades elegidas y a lo largo de distintas franjas horarias (hora local). Es importante notar que el traceroute de linux fue ejecutado con el flag -I, que implica la utilización de la implementación de mensajes ICMP, equivalente a la propia implementación con Scapy. La primera observación interesante que sale a flote de este gráfico es que para todas las franjas salvo las 11hs y 15hs, los valores promediados de ambos programas resultan bastante parecidos. En estas franjas se alternan situaciones en donde una implementación funcionó un poco mejor que otra con lo cual el balance es que se comportan muy parecido. Caso aparte son las dos franjas resaltadas anteriormente: es clara aquí una latencia mucho mayor para las corridas de Python. Esto se debe a que ambos experimentos para esos horarios fueron llevados a cabo desde una oficina donde la red local de origen tiene muchos nodos y tráfico, sobretodo en esos horarios. Esta situación resulta mucho más rica para el análisis del trabajo, puesto que podemos introducir como factor determinante en la performance del RTT a la red origen desde la cual salen los mensajes y a través de la cual deben recepcionarse las respuestas. En cuanto a las diferencias menores en los otros puntos, esto se atribuye a que las corridas de Python fueron ejecutadas desde un equipo en una red particular y las del traceroute de linux en otro entorno hogareño distinto.\\
\indent Más allá de las comparaciones entre implementaciones, así como se tuvo en cuenta el entorno desde el que salen los mensajes, también será determinante el entorno que recibe y da respuesta. Si bien las transmisiones se ven afectadas por todos los lugares por donde pasan, en particular hacemos foco en los extremos del enlace. Por esto, es interesante reparar en cómo varían los RTT de acuerdo a la franja horaria y el huso horario correspondiente en la Universidad destino y como en los momentos de menor actividad en cierto destino para algunos destinos, como Tokio, el RTT se ve disminuído. Otro factor no menos importante es la distancia entre los puntos del enlace, lo cual también se aprecia que impacta en la variación de los tiempos. Asimismo, es importante tener en cuenta que durante una misma ejecución de traceroute los sucesivos paquetes podrían viajar por distintas ramas de la red (eventuales caídas de nodos por ejemplo, o scheduling distinto) lo cual también pondera diferencias en los tiempos.\\
\indent Las \textbf{Figuras~\ref{fig:emp_vs_teo_berk},~\ref{fig:emp_vs_teo_ox},~\ref{fig:emp_vs_teo_tok}}, en cambio, proceden a una visualización puntual del experimento realizado sobre cada Universidad destino, contrastando los RTT de cada implementación de traceroute con el RTT teórico correspondiente según se presentó en la \textbf{Sección 3.1}. Nuevamente, a pesar de la diferencia, esta serie de gráficos invita a reflexionar sobre cuestiones muy interesantes. A simple vista se ve que en los tres casos los RTT empíricos son considerablemente superiores al valor teórico, que se representa como una constante en el tiempo. Pues bien, esta última observación pone en jaque la naturaleza del modelo teórico: dada la topología de una red, que está inmersa en un sistema mundial explotado por las actividades humanas y afectada constantemente por su tráfico y el entorno que la rodea, no parece muy robusta la idea de que un round-trip time sea constante en el tiempo sin tener en cuenta en lo más mínimo estos factores ajenos a la estricta física de los materiales y la energía. Además, se debe recordar que en el cálculo teórico se estimó la distancia entre la ciudad origen y la ciudad que contiene a la universidad destino del enlace de manera aproximadamente lineal, lo cual muy probablemente no se condiga con la verdadera naturaleza física de la ruta. Por último quedan los términos despreciables y algunas asunciones sobre la naturaleza del medio involucrado en la transmisión que aportan su considerable grado de error frente al valor real. Debido a estas razones, los resultados obtenidos no sólo no sorprenden sino que explicitan el abismo entre los prolijos cálculos teóricos y la cruda materialización de auqello que intentan representar. De todas maneras, vemos que se mantiene la tendencia de que a mayor RTT empírico para un destino comparado con otro, mayor RTT teórico (o viceversa, según cómo se lo quiera enfocar).\\
\indent La única excepción a toda esta reflexión sea probablemente el promedio de las ejecuciones de linux para Tokio durante las 15hs (GMT-3) pero 03hs (GMT+9) allá (recordar que para esta franja, las corridas de python se hicieron en un entorno asediado por el tráfico local). Resulta por demás interesante que esta etapa del experimento haya resultado tan cercana al valor teórico, sin embargo notar que sigue estando por encima, lo cual permite generalizar para todos los casos que el RTT teórico es una cota inferior relativamente holgada para los RTT de las ubicaciones involucradas en este trabajo.\\
\indent En el conjunto de \textbf{Figuras~\ref{fig:rtt-horarios-california},~\ref{fig:rtt-horarios-oxford},~\ref{fig:rtt-horarios-tokyo}} se eligió, para cada Universidad, contrastar los valores de RTT para dos franjas específicas a lo largo de las 20 iteraciones realizadas para cada intervalo. Aquí abandonamos los valores de RTT suavizados y promediados para dejar al desnudo la naturaleza variable de las mediciones. Lo ilustrativo de estos gráficos es que podemos apreciar la nube de tiempos de puntos para cada franja, además de poder contrastar dos horarios con una relación particular entre sí (menos - más actividad en origen o destino, etc). En los resultados para la Universidad de Tokio es muy marcada la diferencia entre la performance del envío de mensajes durante las 02 am local (02 pm destino) y las 3pm local (3 am destino). Podemos apreciar que a las 15hs el roundtrip time puede bajar hasta más de la mitad que el visto a las 02hs, momento en el que es mucho más probable que haya gente en la facultad utilizando los equipos y la red, y en la ciudad misma. En la figura~\ref{fig:rtt-horarios-california} se puede ver el caso confeccionado para la Universidad de Berkeley, California. Su huso horario acusa 4hs menos que el de Argentina, por lo tanto, las franjas más diferenciadas son las 11hs (07hs) y 22hs (18hs). Este resultado es un tanto más ambiguo, observándose cierta estabilidad a las 11hs (7hs) con un RTT que está en general alrededor de los 240ms, y un comportamiento más disperso a las 22hs (18hs), momento en el cual se ven RTTs más dilatados (entre 320 y 520ms, aproximadamente), y otros un poco mejores (cerca de los 180 o 200ms). Si se repara en el caso de la Universidad de Oxford, cuyo horario es 4 horas más del local, se verán resultados sutilmente distintos a los de Tokio a pesar de tratarse de una situación similar (oposición de franjas de baja y alta afluencia). Contrario a las expectativas, la figura~\ref{fig:rtt-horarios-oxford} muestra que en la franja de las 15hs (19hs) hay un desempeño mucho más estable que en el de las 22hs (02hs del día siguiente). Dicho desempeño no sólo es más estable, sino que también es mejor (menos RTT en general). Esta es una nueva muestra de que si bien pueden corresponderse algunas hipótesis no muy fuertes, es factible que no se pueda corresponder a la totalidad de los casos, encontrando situaciones de comportamiento dispar como la mencionada, no sólo en términos de medición sino también de expresión del fenómeno.\\
\indent Finalmente, en lo que respecta al análisis y contraste de los RTTs empíricos y teóricos, se presenta la \textbf{Figura~\ref{fig:rtt-normalizado}} donde se intenta poner en fase las distintas mediciones a las universidades para disponer los RTTs en función de una única jornada laboral global (es decir, se transformaron las franjas horarias al valor correspondiente en la localidad destino). En sintonía con la idea planteada en las líneas anteriores, aquí se trató de validar de manera universal la hipótesis de que fuera del horario laboral en cada ubicación geográfica el RTT debería ser menor al medido durante el horario de oficina. Esto se cumple en mayor grado para el caso de Tokio, de manera mucho menos acentuada para Berkeley pero no se corresponde para el caso de Oxford, donde de hecho la gráfica describe síntomas inversos a los esperados. Nuevamente es importante invocar la variabilidad el experimento y de las condiciones en que se toman las muestras, trayendo nuevamente a un primer plano la ruta que compone a un enlace.
